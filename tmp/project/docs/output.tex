\documentclass[12pt,fleqn,leqno]{scrbook}
\usepackage{ae}
%\usepackage{german}
\usepackage{graphics}
\usepackage{graphicx}
\usepackage{amssymb}
\usepackage[latin1]{inputenc}

\usepackage[T1]{fontenc}
\usepackage{listings}
\usepackage{hyperref}
\usepackage{setspace}
\usepackage{makeidx}
\usepackage{alltt} % fuer Mathematisches Zeug innerhalb von verbatim-Bloecken
\usepackage{array}
%\usepackage{bibgerm} % fuer .bib zeug in DE
\usepackage[intoc]{nomencl}
\usepackage{nomencl}
\usepackage[a4paper,left=3.5cm,right=1.5cm,bottom=3cm,top=3.5cm]{geometry}

%\makeindex

\counterwithout{section}{chapter}
\counterwithout{figure}{chapter}

\title{WendzelNNTPd Version 2.2 Documentation}
\author{Steffen Wendzel and contributors}

\begin{document}

\maketitle

%\addcontentsline{toc}{chapter}{Table of Contents}
\tableofcontents

\hypertarget{basic-configuration}{%
\section{Basic Configuration}\label{basic-configuration}}

\protect\hypertarget{Ch:Config}{}{}

This chapter will explain how to configure WendzelNNTPd after
installation.

\textbf{Note:} The configuration file for WendzelNNTPd is named
\emph{/usr/local/etc/wendzelnntpd.conf}. The format of the configuration
file should be self-explanatory and the default configuration file
includes many comments which will help you to understand its content.

\textbf{Note:} On *nix-like operating systems the default installation
path is \emph{/usr/local/*}, i.e., the configuration file of
WendzelNNTPd will be \emph{/usr/local/etc/wendzelnntpd.conf}, and the
binaries will be placed in \emph{/usr/local/sbin}.

\hypertarget{choosing-a-database-engine}{%
\subsection{Choosing a database
engine}\label{choosing-a-database-engine}}

The first and most important step is to choose a database engine. You
can use either SQLite3 (this is the default case and easy to use, but
not suitable for larger systems with many thousand postings or users) or
MySQL (which is the more advaned solution, but also a little bit more
complicated to realize). By default, WendzelNNTPd is configured for
SQLite3 and is ready to run. If you want to keep this setting, you do
not have to read this section.

\hypertarget{modifying-wendzelnntpd.conf}{%
\subsubsection{Modifying
wendzelnntpd.conf}\label{modifying-wendzelnntpd.conf}}

In the configuration file you will find a parameter called
\textbf{database-engine}. You can choose to use either MySQL or SQLite
as the backend storage system by appending either \textbf{sqlite} or
\textbf{mysql}. Experimental support for PostgreSQL can be activiated
with \textbf{postgres}.

\begin{verbatim}
database-engine mysql
\end{verbatim}

If you choose to use MySQL then you will also need to specify the user
and password which WendzelNNTPd must use to connect to the MySQL server.
If your server does not run on localhost or uses a non-default MySQL
port then you will have to modify these values too.

\begin{verbatim}
; Your database hostname (not needed for sqlite3)
database-server 127.0.0.1

; the database connection port (not needed for sqlite3)
; Comment out to use the default port of your database engine
database-port 3306

; Server authentication (not needed for sqlite3)
database-username mysqluser
database-password supercoolpass
\end{verbatim}

\hypertarget{generating-your-database-tables}{%
\subsubsection{Generating your database
tables}\label{generating-your-database-tables}}

Once you have chosen your database backend you will need to create the
database and the required tables.

\hypertarget{sqlite}{%
\paragraph{SQLite}\label{sqlite}}

If you chose SQLite as your database backend then you can skip this step
as running \textbf{make install} does this for you.

\textbf{Note:} The SQLite database file as well as the posting
management files will be stored in \emph{/var/spool/news/wendzelnntpd/}.

\hypertarget{mysql}{%
\paragraph{MySQL}\label{mysql}}

For MySQL, an SQL script file called \emph{mysql\_db\_struct.sql} is
included. It creates the WendzelNNTPd database and all the needed
tables. Use the MySQL console tool to execute the script.

\begin{verbatim}
$ cd /path/to/your/extracted/wendzelnntpd-archive/
$ mysql -u YOUR-USER -p
Enter password:
Welcome to the MySQL monitor.  Commands end with ; or \g.
Your MySQL connection id is 48
Server version: 5.1.37-1ubuntu5.1 (Ubuntu)

Type 'help;' or '\h' for help. Type '\c' to clear the current input statement.

mysql> source mysql\_db\_struct.sql
...
mysql> quit
Bye
\end{verbatim}

\hypertarget{postgresql}{%
\paragraph{PostgreSQL}\label{postgresql}}

Similarly to MySQL, there is a SQL script file
(\emph{postgres\_db\_struct.sql}) to create the WendzelNNTPd database.
Create and setup a new database (and an corresponding user) and use the
\texttt{psql(1)} command line client to load table and function
definitions:

\begin{verbatim}
$ psql --username USER -W wendzelnntpd
wendzelnntpd=> begin;
wendzelnntpd=> \i database/postgres_db_struct.sql
wendzelnntpd=> commit; quit;
\end{verbatim}

\hypertarget{network-settings}{%
\subsection{Network Settings}\label{network-settings}}

For each type of IP address (IPv4 and/or IPv6) you have to define a own
connector. You can find an example for NNTP over port 119 below.

\begin{verbatim}
<connector>
    ;; enables STARTTLS for this port
    ;enable-starttls
    port        119
    listen      127.0.0.1
    ;; configure SSL server certificate (required)
    ;tls-server-certificate "/usr/local/etc/ssl/server.crt"
    ;; configure SSL private key (required)
    ;tls-server-key "/usr/local/etc/ssl/server.key"
    ;; configure SSL CA certificate (required)
    ;tls-ca-certificate "/usr/local/etc/ssl/ca.crt"
    ;; configure TLS ciphers for TLSv1.3
    ;tls-cipher-suites "TLS_AES_128_GCM_SHA256"
    ;; configure TLS ciphers for TLSv1.1 and TLSv1.2
    ;tls-ciphers "ALL:!COMPLEMENTOFDEFAULT:!eNULL"
    ;; configure allowed TLS version (1.0-1.3)
    ;tls-version "1.2-1.3"
    ;; possibility to force the client to authenticate 
    ;;with client certificate (none | optional | require)
    ;tls-verify-client "required"
    ;; define depth for checking client certificate
    ;tls-verify-client-depth 0
    ;; possibility to use certificate revocation list (none | leaf | chain)
    ;tls-crl "none"
    ;tls-crl-file "/usr/local/etc/ssl/ssl.crl"
</connector>
\end{verbatim}

To use dedicated TLS with NNTP (SNNTP) you can define another connector.
The example below is for SNNTP over port 563.

\begin{verbatim}
<connector>
    ;; enables TLS for this port
    ;enable-tls
    port        563
    listen      127.0.0.1
    ;; configure SSL server certificate (required)
    ;tls-server-certificate "/usr/local/etc/ssl/server.crt"
    ;; configure SSL private key (required)
    ;tls-server-key "/usr/local/etc/ssl/server.key"
    ;; configure SSL CA certificate (required)
    ;tls-ca-certificate "/usr/local/etc/ssl/ca.crt"
    ;; configure TLS ciphers for TLSv1.3
    ;tls-cipher-suites "TLS_AES_128_GCM_SHA256"
    ;; configure TLS ciphers for TLSv1.1 and TLSv1.2
    ;tls-ciphers "ALL:!COMPLEMENTOFDEFAULT:!eNULL"
    ;; configure allowed TLS version (1.0-1.3)
    ;tls-version "1.2-1.3"
    ;; possibility to force the client to authenticate 
    ;;with client certificate (none | optional | require)
    ;tls-verify-client "required"
    ;; define depth for checking client certificate
    ;tls-verify-client-depth 0
    ;; possibility to use certificate revocation list (none | leaf | chain)
    ;tls-crl "none"
    ;tls-crl-file "/usr/local/etc/ssl/ssl.crl"
</connector>
\end{verbatim}

The configuration options \emph{tls-server-certificate},
\emph{tls-server-key} and \emph{tls-ca-certificate} are required for
using TLS or STARTTLS with NNTP. All other TLS-related options are
optional. More examples are in the existing \emph{wendzelnntpd.conf}
file.

\hypertarget{setting-the-allowed-size-of-postings}{%
\subsection{Setting the Allowed Size of
Postings}\label{setting-the-allowed-size-of-postings}}

To change the maximum size of a post to be accepted by the server,
change the variable \textbf{max-size-of-postings}. The value must be set
in Bytes and the default value is 20971520 (20 MBytes).

\begin{verbatim}
max-size-of-postings 20971520
\end{verbatim}

\hypertarget{verbose-mode}{%
\subsection{Verbose Mode}\label{verbose-mode}}

If you have any problems running WendzelNNTPd or if you simply want more
information about what is happening, you can uncomment the
\textbf{verbose-mode} line.

\begin{verbatim}
; Uncomment 'verbose-mode' if you want to find errors or if you
; have problems with the logging subsystem. All log strings are
; written to stderr too, if verbose-mode is set. Additionally all
; commands sent by clients are written to stderr too (but not to
; logfile)
verbose-mode
\end{verbatim}

\hypertarget{security-settings}{%
\subsection{Security Settings}\label{security-settings}}

\hypertarget{authentication-and-access-control-lists-acl}{%
\subsubsection{Authentication and Access Control Lists
(ACL)}\label{authentication-and-access-control-lists-acl}}

WendzelNNTPd contains an extensive access control subsystem. If you want
to only allow authenticated users to access the server, you should
uncomment \textbf{use-authentication}. This gives every authenticated
user access to each newsgroup.

\begin{verbatim}
; Activate authentication
use-authentication
\end{verbatim}

If you need a slightly more advanced authentication system, you can
activate Access Control Lists (ACL) by uncommenting \textbf{use-acl}.
This activates the support for Role-based ACL too.

\begin{verbatim}
; If you activated authentication, you can also activate access
; control lists (ACL)
use-acl
\end{verbatim}

\hypertarget{anonymized-message-id}{%
\subsubsection{Anonymized Message-ID}\label{anonymized-message-id}}

By default, WendzelNNTPd makes a user's hostname or IP address part of
new message IDs when a user sends a post using the NNTP POST command. If
you do not want that, you can force WendzelNNTPd not to do so by
uncommenting \textbf{enable-anonym-mids}, which enables anonymized
message IDs.

\begin{verbatim}
; This prevents that IPs or Hostnames will become part of the
; message ID generated by WendzelNNTPd what is the default case.
; Uncomment it to enable this feature.
enable-anonym-mids
\end{verbatim}

\hypertarget{changing-the-default-salt-for-password-hashing}{%
\subsubsection{Changing the Default Salt for Password
Hashing}\label{changing-the-default-salt-for-password-hashing}}

When uncommenting the keyword \textbf{hash-salt}, the default salt value
that is used to enrich the password hashes can be changed. Please note
that you have to define the salt \emph{before} you set-up the first
password since it will otherwise be stored hashed using an old salt,
rendering it unusable. For this reason, it is necessary to define your
salt right after running \textbf{make install} (or at least before the
first creation of NNTP user accounts).

\begin{verbatim}
; This keyword defines a salt to be used in conjunction with the
; passwords to calculate the cryptographic hashes. The salt must
; be in the form [a-zA-Z0-9.:\/-_]+.
; ATTENTION: If you change the salt after passwords have been
; stored, they will be rendered invalid! If you comment out
; hash-salt, then the default hash salt defined in the source
; code will be used.
hash-salt 0.hG4//3baA-::_\
\end{verbatim}

WendzelNNTPd applies the SHA-2 hash algorithm using a 256 bit hash
value. Please also note that the final hash is calculated using a string
that combines salt, username and password as an input to prevent
password-identification attacks when an equal password is used by
multiple users. However, utilizing the username is less secure than
having a completely separate salt for every password.\footnote{Some
  *nix-like systems may have a different range of privileged ports.}

\hypertarget{encrypted-communication-tls}{%
\subsubsection{Encrypted communication
(TLS)}\label{encrypted-communication-tls}}

Please look at section \protect\hyperlink{network-settings}{1.2} when
you want to use encryption over TLS.

\hypertarget{development}{%
\section{Development}\label{development}}

For development purposes you can start WendzelNNTPd on your host system.
If you are using an unsupported operating system or just not run
WenzdelNNTPd on your host system, you can use the provided Dockerfiles
to run WendzelNNTPd in a Docker container.

\hypertarget{initial-setup}{%
\subsubsection{Initial setup}\label{initial-setup}}

When you are on a UNIX-based system (like macOS) you can use the
provided make commands:

\begin{verbatim}
$ make docker-build
$ make docker-run
\end{verbatim}

To stop the Docker container you can use the following command:

\begin{verbatim}
$ make docker-stop
\end{verbatim}

If you are not on a UNIX-based system (like Windows) use the following
native docker commands:

\begin{verbatim}
$ docker build -f ./docker/Dockerfile -t wendzelnntpd:latest .
$ docker run --name wendzelnntpd --rm -it -p 118:118 -p 119:119 -p 563:563 -p 564:564 -v ${PWD}:/wendzelnntpd -v wendzelnntpd_data:/var/spool/news/wendzelnntpd wendzelnntpd:latest
\end{verbatim}

To stop the Docker container you can use the following command:

\begin{verbatim}
$ docker stop wendzelnntpd
\end{verbatim}

\hypertarget{test-new-code}{%
\subsubsection{Test new code}\label{test-new-code}}

The container is build without code included. The code is automatically
mounted as volume into the container. After each change of source code,
the application is compiled again.

\hypertarget{introduction}{%
\section{Introduction}\label{introduction}}

WendzelNNTPd is a tiny but easy to use Usenet server (NNTP server) for
Linux, *nix and BSD. The server is written in C. For security reasons,
it is compiled with stack smashing protection by default, if your
compiler supports that feature.

\hypertarget{features}{%
\subsection{Features}\label{features}}

\hypertarget{license}{%
\subsubsection{License}\label{license}}

WendzelNNTPd uses the GPLv3 license.

\hypertarget{database-abstraction-layer}{%
\subsubsection{Database Abstraction
Layer}\label{database-abstraction-layer}}

The server contains a database abstraction layer. Currently supported
database systems are SQlite3 and MySQL (and experimental PostgreSQL
support). New databases can be easily added.

\hypertarget{security}{%
\subsubsection{Security}\label{security}}

WendzelNNTPd contains different security features, the most important
features are probably Access Control Lists (ACLs) and the Role Based
Access Control (RBAC) system. ACL and RBAC are described in a own
chapter. WendzelNNTPd was probably the first Usenet server with support
for RBAC.

Another feature that was introduced by WendzelNNTPd (and later adopted
by other servers) are so-called ``invisible newsgroups'': If access
control is activated, a user without permission to access the newsgroup
will not be able to see the existence of the newsgroup. In case the user
knows about the existence of the newsgroup nevertheless, he will not be
able to post to or read from the newsgroup.

However, \textbf{please note} that the salting for password hashing is
using SHA-256, but with a global user-definable salt that is
concatenated with the username and password, rendering it less secure
than using unique random hashes per password. WendzelNNTPd does support
TLS v1.0 to v1.3 including STARTTLS!

\hypertarget{auto-prevention-of-double-postings}{%
\subsubsection{Auto-prevention of
double-postings}\label{auto-prevention-of-double-postings}}

In case a user sends a posting that lists the same newsgroup multiple
times within one post command's ``Newsgroups:'' header tag, the server
will add it only once to that newsgroup to save memory on the server and
the time of the readers.

\hypertarget{ipv6}{%
\subsubsection{IPv6}\label{ipv6}}

WendzelNNTPd supports IPv6. The server can listen on multiple IP
addresses as well as multiple ports.

\hypertarget{why-this-is-not-a-perfect-usenet-server}{%
\subsubsection{Why this is not a perfect Usenet
server}\label{why-this-is-not-a-perfect-usenet-server}}

WendzelNNTPd does not implement all NNTP commands, but the (most)
important ones. Another problem is that the regular expression library
used is not 100\% compatible with the NNTP matching in commands like
``XGTITLE''. Another limitation is that WendzelNNTPd cannot share
messages with other NNTP servers.

\hypertarget{contribute}{%
\subsection{Contribute}\label{contribute}}

See the \emph{CONTRIBUTE} file in the tarball.

\hypertarget{history}{%
\subsection{History}\label{history}}

The project started in 2004 under the name Xyria:cdpNNTPd, as part of
the Xyria project that also contained a fast DNS server, called
Xyria:DNSd. In 2007, I renamed it to WendzelNNTPd and stopped
development of Xyria:DNSd. Version 1.0.0 was released in 2007, version
2.0.0 in 2011. Since then I have primarily fixed reported bugs and added
minor features but the software is still maintained and smaller
advancements can still be expected. A detailed history can be found in
the \emph{HISTORY} file in the tarball. Fortunately, several people
contributed to the code and documentation, see \emph{AUTHORS} file.

\hypertarget{installation}{%
\section{Installation}\label{installation}}

This chapter provides a guide on how to install WendzelNNTPd 2.x.

\hypertarget{linuxnixbsd}{%
\subsection{Linux/*nix/BSD}\label{linuxnixbsd}}

To install WendzelNNTPd from source you need to download the provided
archive file (e.g., \emph{wendzelnntpd-2.0.0.tar.gz}) file.\footnote{Some
  *nix-like systems may have a different range of privileged ports.}
Extract it and run \textbf{./configure}. Please note that configure
indicates missing libraries and packages that you may first need to
install using the package system of your operating system.

\begin{verbatim}
$ tar -xzf wendzelnntpd-2.0.0.tgz
$ cd wendzelnntpd
$ ./configure
...
\end{verbatim}

\textbf{Please Note:} \emph{If you wish to compile WITHOUT MySQL or
WITHOUT SQlite support}, then run \textbf{MYSQL=NO ./configure} or
\textbf{SQLITE=NO ./configure}, respectively.

~

\textbf{Please Note:} \emph{For FreeBSD/OpenBSD/NetBSD/*BSD: There is no
MySQL support; you need to use SQLite (i.e., run
\texttt{MYSQL=NO\ ./configure}). Run \texttt{configure} as well as
\texttt{make} in the \texttt{bash} shell (under some BSDs you first need
to install \texttt{bash}).}

~

\textbf{Please Note:} \emph{If you wish to compile WITHOUT TLS support},
then run \textbf{TLS=NO ./configure}.

~

After \texttt{configure} has finished, run \textbf{make}:

\begin{verbatim}
$ make
...
\end{verbatim}

\hypertarget{generating-ssl-certifiates}{%
\subparagraph*{Generating SSL
certifiates}\label{generating-ssl-certifiates}}
\addcontentsline{toc}{subparagraph}{Generating SSL certifiates}

If you want to generate SSL certificates you can use the helper script:

\begin{verbatim}
$ sudo ./create_certificate \
    --environment letsencrypt \
    --email <YOUR-EMAIL> \\
    --domain <YOUR-DOMAIN>
\end{verbatim}

For the parameter \texttt{-\/-environment}, ``\emph{local}'' is also a
valid value. In that case, the certificate is generated only for usage
on localhost and is self-signed. After generating the certificate you
have to adjust \emph{wendzelnntpd.conf} (check
Section~\protect\hyperlink{network-settings}{3.2}) to activate TLS
(configuration option \emph{enable-tls})). The paths for certificate and
server key can stay as they are.

~

\hypertarget{installing-wendzelnntpd}{%
\subparagraph*{Installing WendzelNNTPd}\label{installing-wendzelnntpd}}
\addcontentsline{toc}{subparagraph}{Installing WendzelNNTPd}

To install WendzelNNTPd on your system, you need superuser access. Run
\textbf{make install} to install it to the default location
\emph{/usr/local/*}.

\begin{verbatim}
$ sudo make install
\end{verbatim}

\textbf{Please Note (Upgrades):} Run \textbf{sudo make upgrade} instead
of \textbf{sudo make install} for an upgrade. Please
cf.~Section~\protect\hyperlink{Ch:Upgrade}{{[}Ch:Upgrade{]}}
(Upgrading).

\textbf{Please Note (MySQL):} \emph{If you plan to run MySQL}, then no
database was set up during `make install'. Please refer to
Section~\protect\hyperlink{Ch:Config}{{[}Ch:Config{]}} (Basic
Configuration) to learn how to generate the MySQL database.

\hypertarget{init-script-for-automatic-startup}{%
\subsubsection{Init Script for Automatic
Startup}\label{init-script-for-automatic-startup}}

There is an init script in the directory scripts/startup. It uses the
usual parameters like ``start'', ``stop'' and ``restart''.

\hypertarget{unofficial-note-mac-os-x}{%
\subsection{Unofficial Note: Mac OS X}\label{unofficial-note-mac-os-x}}

A user reported WendzelNNTPd-2.0.0 is installable under Mac OS X 10.5.8.
The only necessary change was to add the flag
``\texttt{-arch\ x86\_64}'' to compile the code on a 64 bit system.
However, I never tried to compile WendzelNNTPd on a Mac.

\hypertarget{windows}{%
\subsection{Windows}\label{windows}}

Not supported.

\hypertarget{basic-configuration-1}{%
\section{Basic Configuration}\label{basic-configuration-1}}

\protect\hypertarget{Ch:Config}{}{}

This chapter will explain how to configure WendzelNNTPd after
installation.

\textbf{Note:} The configuration file for WendzelNNTPd is named
\emph{/usr/local/etc/wendzelnntpd.conf}. The format of the configuration
file should be self-explanatory and the default configuration file
includes many comments which will help you to understand its content.

\textbf{Note:} On *nix-like operating systems the default installation
path is \emph{/usr/local/*}, i.e., the configuration file of
WendzelNNTPd will be \emph{/usr/local/etc/wendzelnntpd.conf}, and the
binaries will be placed in \emph{/usr/local/sbin}.

\hypertarget{choosing-a-database-engine-1}{%
\subsection{Choosing a database
engine}\label{choosing-a-database-engine-1}}

The first and most important step is to choose a database engine. You
can use either SQLite3 (this is the default case and easy to use, but
not suitable for larger systems with many thousand postings or users) or
MySQL (which is the more advaned solution, but also a little bit more
complicated to realize). By default, WendzelNNTPd is configured for
SQLite3 and is ready to run. If you want to keep this setting, you do
not have to read this section.

\hypertarget{modifying-wendzelnntpd.conf-1}{%
\subsubsection{Modifying
wendzelnntpd.conf}\label{modifying-wendzelnntpd.conf-1}}

In the configuration file you will find a parameter called
\textbf{database-engine}. You can choose to use either MySQL or SQLite
as the backend storage system by appending either \textbf{sqlite} or
\textbf{mysql}. Experimental support for PostgreSQL can be activiated
with \textbf{postgres}.

\begin{verbatim}
database-engine mysql
\end{verbatim}

If you choose to use MySQL then you will also need to specify the user
and password which WendzelNNTPd must use to connect to the MySQL server.
If your server does not run on localhost or uses a non-default MySQL
port then you will have to modify these values too.

\begin{verbatim}
; Your database hostname (not needed for sqlite3)
database-server 127.0.0.1

; the database connection port (not needed for sqlite3)
; Comment out to use the default port of your database engine
database-port 3306

; Server authentication (not needed for sqlite3)
database-username mysqluser
database-password supercoolpass
\end{verbatim}

\hypertarget{generating-your-database-tables-1}{%
\subsubsection{Generating your database
tables}\label{generating-your-database-tables-1}}

Once you have chosen your database backend you will need to create the
database and the required tables.

\hypertarget{sqlite-1}{%
\paragraph{SQLite}\label{sqlite-1}}

If you chose SQLite as your database backend then you can skip this step
as running \textbf{make install} does this for you.

\textbf{Note:} The SQLite database file as well as the posting
management files will be stored in \emph{/var/spool/news/wendzelnntpd/}.

\hypertarget{mysql-1}{%
\paragraph{MySQL}\label{mysql-1}}

For MySQL, an SQL script file called \emph{mysql\_db\_struct.sql} is
included. It creates the WendzelNNTPd database and all the needed
tables. Use the MySQL console tool to execute the script.

\begin{verbatim}
$ cd /path/to/your/extracted/wendzelnntpd-archive/
$ mysql -u YOUR-USER -p
Enter password:
Welcome to the MySQL monitor.  Commands end with ; or \g.
Your MySQL connection id is 48
Server version: 5.1.37-1ubuntu5.1 (Ubuntu)

Type 'help;' or '\h' for help. Type '\c' to clear the current input statement.

mysql> source mysql\_db\_struct.sql
...
mysql> quit
Bye
\end{verbatim}

\hypertarget{postgresql-1}{%
\paragraph{PostgreSQL}\label{postgresql-1}}

Similarly to MySQL, there is a SQL script file
(\emph{postgres\_db\_struct.sql}) to create the WendzelNNTPd database.
Create and setup a new database (and an corresponding user) and use the
\texttt{psql(1)} command line client to load table and function
definitions:

\begin{verbatim}
$ psql --username USER -W wendzelnntpd
wendzelnntpd=> begin;
wendzelnntpd=> \i database/postgres_db_struct.sql
wendzelnntpd=> commit; quit;
\end{verbatim}

\hypertarget{network-settings-1}{%
\subsection{Network Settings}\label{network-settings-1}}

For each type of IP address (IPv4 and/or IPv6) you have to define a own
connector. You can find an example for NNTP over port 119 below.

\begin{verbatim}
<connector>
    ;; enables STARTTLS for this port
    ;enable-starttls
    port        119
    listen      127.0.0.1
    ;; configure SSL server certificate (required)
    ;tls-server-certificate "/usr/local/etc/ssl/server.crt"
    ;; configure SSL private key (required)
    ;tls-server-key "/usr/local/etc/ssl/server.key"
    ;; configure SSL CA certificate (required)
    ;tls-ca-certificate "/usr/local/etc/ssl/ca.crt"
    ;; configure TLS ciphers for TLSv1.3
    ;tls-cipher-suites "TLS_AES_128_GCM_SHA256"
    ;; configure TLS ciphers for TLSv1.1 and TLSv1.2
    ;tls-ciphers "ALL:!COMPLEMENTOFDEFAULT:!eNULL"
    ;; configure allowed TLS version (1.0-1.3)
    ;tls-version "1.2-1.3"
    ;; possibility to force the client to authenticate 
    ;;with client certificate (none | optional | require)
    ;tls-verify-client "required"
    ;; define depth for checking client certificate
    ;tls-verify-client-depth 0
    ;; possibility to use certificate revocation list (none | leaf | chain)
    ;tls-crl "none"
    ;tls-crl-file "/usr/local/etc/ssl/ssl.crl"
</connector>
\end{verbatim}

To use dedicated TLS with NNTP (SNNTP) you can define another connector.
The example below is for SNNTP over port 563.

\begin{verbatim}
<connector>
    ;; enables TLS for this port
    ;enable-tls
    port        563
    listen      127.0.0.1
    ;; configure SSL server certificate (required)
    ;tls-server-certificate "/usr/local/etc/ssl/server.crt"
    ;; configure SSL private key (required)
    ;tls-server-key "/usr/local/etc/ssl/server.key"
    ;; configure SSL CA certificate (required)
    ;tls-ca-certificate "/usr/local/etc/ssl/ca.crt"
    ;; configure TLS ciphers for TLSv1.3
    ;tls-cipher-suites "TLS_AES_128_GCM_SHA256"
    ;; configure TLS ciphers for TLSv1.1 and TLSv1.2
    ;tls-ciphers "ALL:!COMPLEMENTOFDEFAULT:!eNULL"
    ;; configure allowed TLS version (1.0-1.3)
    ;tls-version "1.2-1.3"
    ;; possibility to force the client to authenticate 
    ;;with client certificate (none | optional | require)
    ;tls-verify-client "required"
    ;; define depth for checking client certificate
    ;tls-verify-client-depth 0
    ;; possibility to use certificate revocation list (none | leaf | chain)
    ;tls-crl "none"
    ;tls-crl-file "/usr/local/etc/ssl/ssl.crl"
</connector>
\end{verbatim}

The configuration options \emph{tls-server-certificate},
\emph{tls-server-key} and \emph{tls-ca-certificate} are required for
using TLS or STARTTLS with NNTP. All other TLS-related options are
optional. More examples are in the existing \emph{wendzelnntpd.conf}
file.

\hypertarget{setting-the-allowed-size-of-postings-1}{%
\subsection{Setting the Allowed Size of
Postings}\label{setting-the-allowed-size-of-postings-1}}

To change the maximum size of a post to be accepted by the server,
change the variable \textbf{max-size-of-postings}. The value must be set
in Bytes and the default value is 20971520 (20 MBytes).

\begin{verbatim}
max-size-of-postings 20971520
\end{verbatim}

\hypertarget{verbose-mode-1}{%
\subsection{Verbose Mode}\label{verbose-mode-1}}

If you have any problems running WendzelNNTPd or if you simply want more
information about what is happening, you can uncomment the
\textbf{verbose-mode} line.

\begin{verbatim}
; Uncomment 'verbose-mode' if you want to find errors or if you
; have problems with the logging subsystem. All log strings are
; written to stderr too, if verbose-mode is set. Additionally all
; commands sent by clients are written to stderr too (but not to
; logfile)
verbose-mode
\end{verbatim}

\hypertarget{security-settings-1}{%
\subsection{Security Settings}\label{security-settings-1}}

\hypertarget{authentication-and-access-control-lists-acl-1}{%
\subsubsection{Authentication and Access Control Lists
(ACL)}\label{authentication-and-access-control-lists-acl-1}}

WendzelNNTPd contains an extensive access control subsystem. If you want
to only allow authenticated users to access the server, you should
uncomment \textbf{use-authentication}. This gives every authenticated
user access to each newsgroup.

\begin{verbatim}
; Activate authentication
use-authentication
\end{verbatim}

If you need a slightly more advanced authentication system, you can
activate Access Control Lists (ACL) by uncommenting \textbf{use-acl}.
This activates the support for Role-based ACL too.

\begin{verbatim}
; If you activated authentication, you can also activate access
; control lists (ACL)
use-acl
\end{verbatim}

\hypertarget{anonymized-message-id-1}{%
\subsubsection{Anonymized Message-ID}\label{anonymized-message-id-1}}

By default, WendzelNNTPd makes a user's hostname or IP address part of
new message IDs when a user sends a post using the NNTP POST command. If
you do not want that, you can force WendzelNNTPd not to do so by
uncommenting \textbf{enable-anonym-mids}, which enables anonymized
message IDs.

\begin{verbatim}
; This prevents that IPs or Hostnames will become part of the
; message ID generated by WendzelNNTPd what is the default case.
; Uncomment it to enable this feature.
enable-anonym-mids
\end{verbatim}

\hypertarget{changing-the-default-salt-for-password-hashing-1}{%
\subsubsection{Changing the Default Salt for Password
Hashing}\label{changing-the-default-salt-for-password-hashing-1}}

When uncommenting the keyword \textbf{hash-salt}, the default salt value
that is used to enrich the password hashes can be changed. Please note
that you have to define the salt \emph{before} you set-up the first
password since it will otherwise be stored hashed using an old salt,
rendering it unusable. For this reason, it is necessary to define your
salt right after running \textbf{make install} (or at least before the
first creation of NNTP user accounts).

\begin{verbatim}
; This keyword defines a salt to be used in conjunction with the
; passwords to calculate the cryptographic hashes. The salt must
; be in the form [a-zA-Z0-9.:\/-_]+.
; ATTENTION: If you change the salt after passwords have been
; stored, they will be rendered invalid! If you comment out
; hash-salt, then the default hash salt defined in the source
; code will be used.
hash-salt 0.hG4//3baA-::_\
\end{verbatim}

WendzelNNTPd applies the SHA-2 hash algorithm using a 256 bit hash
value. Please also note that the final hash is calculated using a string
that combines salt, username and password as an input to prevent
password-identification attacks when an equal password is used by
multiple users. However, utilizing the username is less secure than
having a completely separate salt for every password.\footnote{Patches
  are appreciated!}

\hypertarget{encrypted-communication-tls-1}{%
\subsubsection{Encrypted communication
(TLS)}\label{encrypted-communication-tls-1}}

Please look at section \protect\hyperlink{network-settings}{3.2} when
you want to use encryption over TLS.

\hypertarget{starting-and-running-wendzelnntpd}{%
\section{Starting and Running
WendzelNNTPd}\label{starting-and-running-wendzelnntpd}}

\hypertarget{starting-the-service}{%
\subsection{Starting the Service}\label{starting-the-service}}

Once your WendzelNNTPd installation has been configured, you can run the
server (in the default case you need superuser access to do that since
this is required to bind WendzelNNTPd to the default NNTP port 119) by
starting \textbf{/usr/local/sbin/wendzelnntpd}.

\begin{verbatim}
$ /usr/local/sbin/wendzelnntpd 
WendzelNNTPd: version 2.0.7 'Berlin'  - (Oct
26 2015 14:10:20 #2544) is ready.
\end{verbatim}

\textbf{Note (Daemon Mode):} If you want to run WendzelNNTPd as a
background daemon process on *nix-like operating systems, you should use
the parameter \textbf{-d}.

\hypertarget{stopping-and-restarting-the-service}{%
\subsection{Stopping and Restarting the
Service}\label{stopping-and-restarting-the-service}}

The server can be stopped by terminating its process:

\begin{verbatim}
$ pkill wendzelnntpd
\end{verbatim}

The server has a handler for the termination signal that allows to
safely shutdown using \textbf{pkill} or \textbf{kill}.

To restart the service, terminate and start the service.

\hypertarget{automating-start-stop-and-restart}{%
\subsubsection{Automating Start, Stop, and
Restart}\label{automating-start-stop-and-restart}}

The script \textbf{init.d\_script} in the directory
\emph{scripts/startup} of the tarball can be used to start, restart, and
stop WendzelNNTPd. It is a standard \emph{init.d} script for Linux
operating systems that can usually be copied to \emph{/etc/init.d} (it
must be executable).

\begin{verbatim}
$ cp scripts/startup/init.d_script /etc/init.d/wendzelnntpd
$ chmod +x /etc/init.d/wendzelnntpd
\end{verbatim}

\textbf{Note:} Please note that some operating systems use different
directories than \emph{/etc/init.d} or other startup script formats. In
such cases, the script works nevertheless but should simply be installed
to \emph{/usr/local/sbin} instead.

To start, stop, and restart WendzelNNTPd, the following commands can be
used afterwards:

\begin{verbatim}
$ /etc/init.d/wendzelnntpd start
Starting WendzelNNTPd ... done.
WendzelNNTPd: version 2.0.7 'Berlin'  - (Oct
26 2015 14:10:20 #2544) is ready.

$ /etc/init.d/wendzelnntpd restart
Stopping WendzelNNTPd ... done.
Starting WendzelNNTPd ... done.
WendzelNNTPd: version 2.0.7 'Berlin'  - (Oct
26 2015 14:10:20 #2544) is ready.

$ /etc/init.d/wendzelnntpd stop
Stopping WendzelNNTPd ... done.
\end{verbatim}

\hypertarget{administration-tool-wendzelnntpadm}{%
\subsection{Administration Tool
`wendzelnntpadm'}\label{administration-tool-wendzelnntpadm}}

Use the command line tool \textbf{wendzelnntpadm} to configure users,
roles and newsgroups of your WendzelNNTPd installation. To get an
overview of supported commands, run ``wendzelnntpadm help'':

\begin{verbatim}
$ wendzelnntpadm help
usage: wendzelnntpd <command> [parameters]
*** Newsgroup Administration:
 <listgroups>
 <addgroup | modgroup> <newsgroup> <posting-allowed-flag (y/n)>
 <delgroup> <newsgroup>
*** User Administration:
 <listusers>
 <adduser> <username> [<password>]
 <deluser> <username>
*** ACL (Access Control List) Administration:
 <listacl>
 <addacluser | delacluser> <username> <newsgroup>
 <addaclrole | delaclrole> <role>
 <rolegroupconnect | rolegroupdisconnect> <role> <newsgroup>
 <roleuserconnect | roleuserdisconnect> <role> <username>
\end{verbatim}

\hypertarget{creatinglistingdeleting-newsgroups}{%
\subsection{Creating/Listing/Deleting
Newsgroups}\label{creatinglistingdeleting-newsgroups}}

You can either list, create or delete newsgroups using
\textbf{wendzelnntpadm}.

\hypertarget{listing-existing-newsgroups}{%
\subsubsection{Listing existing
newsgroups}\label{listing-existing-newsgroups}}

\begin{verbatim}
$ wendzelnntpadm listgroups
Newsgroup, Low-, High-Value, Posting-Flag
-----------------------------------------
alt.test 10 1 y
mgmt.talk 1 1 y
secret.project-x 20 1 y
done.
\end{verbatim}

\hypertarget{creating-a-new-newsgroup}{%
\subsubsection{Creating a new
newsgroup}\label{creating-a-new-newsgroup}}

To create a new newsgroup run the following command:

\begin{verbatim}
$ wendzelnntpadm addgroup my.cool.group y
Newsgroup my.cool.group does not exist. Creating new group.
done.
\end{verbatim}

You can also change the ``posting allowed'' flag of a newsgroup:

\begin{verbatim}
$ wendzelnntpadm modgroup my.cool.group y
Newsgroup my.cool.group exists: okay.
done.
$ wendzelnntpadm modgroup my.cool.group n
Newsgroup my.cool.group exists: okay.
done.
\end{verbatim}

\hypertarget{deleting-a-newsgroup}{%
\subsubsection{Deleting a newsgroup}\label{deleting-a-newsgroup}}

\begin{verbatim}
$ wendzelnntpadm delgroup my.cool.group
Newsgroup my.cool.group exists: okay.
Clearing association class ... done
Clearing ACL associations of newsgroup my.cool.group... done
Clearing ACL role associations of newsgroup my.cool.group... done
Deleting newsgroup my.cool.group itself ... done
Cleanup: Deleting postings that do not belong to an existing newsgroup ... done
done.
\end{verbatim}

\hypertarget{user-accounts-administration}{%
\subsection{User Accounts
Administration}\label{user-accounts-administration}}

The easiest way to give only some people access to your server is to
create user accounts (please make sure you activated authentication in
your configuration file). You can add, delete and list all users.

\hypertarget{listing-users-and-passwords}{%
\subsubsection{Listing Users (and
Passwords)}\label{listing-users-and-passwords}}

This command always prints the (hashed) password of the users:

\begin{verbatim}
$ wendzelnntpadm listusers
Username, Password
------------------
developer1, wegerhgrhtrthjtzj
developer2, erghnrehhnht
manager1, wegergergrhth
manager2, thnthnrothnht
swendzel, lalalegergreg
swendzel2, 94j5z5jh5th
swendzel3, lalalalala
swendzel4, wegwegwegwegweg
done.
\end{verbatim}

\hypertarget{creating-a-new-user}{%
\subsubsection{Creating a new user}\label{creating-a-new-user}}

You can either enter the password as additional parameter (useful for
scripts that create users automatically) ...

\begin{verbatim}
$ wendzelnntpadm adduser UserName HisPassWord
User UserName does currently not exist: okay.
done.
\end{verbatim}

... or you can type it using the prompt (in this case the input is
shadowed):

\begin{verbatim}
$ wendzelnntpadm adduser UserName2
Enter new password for this user (max. 100 chars):
User UserName2 does currently not exist: okay.
done.
\end{verbatim}

\textbf{Please Note:} A password must include at least 8 characters and
may not include more than 100 characters.

\hypertarget{deleting-an-existing-user}{%
\subsubsection{Deleting an existing
user}\label{deleting-an-existing-user}}

\begin{verbatim}
$ wendzelnntpadm deluser UserName2
User UserName2 exists: okay.
Clearing ACL associations of user UserName2... done
Clearing ACL role associations of user UserName2... done
Deleting user UserName2 from database ... done
done.
\end{verbatim}

\hypertarget{access-control-list-administration-in-case-the-standard-nntp-authentication-is-not-enough}{%
\subsection{Access Control List Administration (in case the standard
NNTP authentication is not
enough)}\label{access-control-list-administration-in-case-the-standard-nntp-authentication-is-not-enough}}

Welcome to the advanced part of WendzelNNTPd. WendzelNNTPd includes a
powerful role based access control system. You can either only use
normal access control lists where you can configure which user will have
access to which newsgroup. Or you can use the advanced role system: You
can add users to roles (e.g., the user ``boss99'' to the role
``management'') and give a role access to a group (e.g., role
``management'' shall have access to ``discuss.management'').

\textbf{Note:} Please note that you must activate the ACL feature in
your configuration file to use it.

\textbf{Note:} To see \emph{all} data related to the ACL subsystem of
your WendzelNNTPd installation, simply use ``wendzelnntpadm listacl''.

\hypertarget{invisible-newsgroups}{%
\subsubsection{Invisible Newsgroups}\label{invisible-newsgroups}}

WendzelNNTPd includes a feature called ``Invisible Newsgroups'' which
means that a user without access to a newsgroup will neither see the
newsgroup in the list of newsgroups, nor will he be able to post to such
a newsgroup or will be able to read it.

\hypertarget{simple-access-control}{%
\subsubsection{Simple Access Control}\label{simple-access-control}}

We start with the simple access control component where you can define
which user will have access to which newsgroup.

\hypertarget{giving-a-user-access-to-a-newsgroup}{%
\paragraph{Giving a user access to a
newsgroup}\label{giving-a-user-access-to-a-newsgroup}}

\begin{verbatim}
$ wendzelnntpadm addacluser swendzel alt.test
User swendzel exists: okay.
Newsgroup alt.test exists: okay.
done.
$ wendzelnntpadm listacl
List of roles in database:
Roles
-----

Connections between users and roles:
Role, User
----------

Username, Has access to group
-----------------------------
swendzel, alt.test

Role, Has access to group
-------------------------
done.
\end{verbatim}

\hypertarget{removing-a-users-access-to-a-newsgroup}{%
\paragraph{Removing a user's access to a
newsgroup}\label{removing-a-users-access-to-a-newsgroup}}

\begin{verbatim}
$ wendzelnntpadm delacluser swendzel alt.test
User swendzel exists: okay.
Newsgroup alt.test exists: okay.
done.
\end{verbatim}

\hypertarget{adding-and-removing-acl-roles}{%
\subsubsection{Adding and Removing ACL
Roles}\label{adding-and-removing-acl-roles}}

If you have many users, some of them should have access to the same
newsgroup (e.g., the developers of a new system should have access to
the development newsgroup of the system). With roles you do not have to
give every user explicit access to such a group. Instead you add the
users to a role and give the role access to the group. (One advantage is
that you can easily give the complete role access to another group with
only one command instead of adding each of its users to the list of
people who have access to the new group).

In the following examples, we give the users ``developer1'',
``developer2'', and ``developer3'' access to the development role of
``project-x'' and connect their role to the newsgroups
``project-x.discussion'' and ``project-x.support''. To do so, we create
the three users and the two newsgroups first:

\begin{verbatim}
$ wendzelnntpadm adduser developer1
Enter new password for this user (max. 100 chars):
User developer1 does currently not exist: okay.
done.
$ wendzelnntpadm adduser developer2
Enter new password for this user (max. 100 chars):
User developer2 does currently not exist: okay.
done.
$ wendzelnntpadm adduser developer3
Enter new password for this user (max. 100 chars):
User developer3 does currently not exist: okay.
done.

$ wendzelnntpadm addgroup project-x.discussion y
Newsgroup project-x.discussion does not exist. Creating new group.
done.
$ wendzelnntpadm addgroup project-x.support y
Newsgroup project-x.support does not exist. Creating new group.
done.
\end{verbatim}

\hypertarget{creating-an-acl-role}{%
\paragraph{Creating an ACL Role}\label{creating-an-acl-role}}

Before you can add users to a role and before you can connect a role to
a newsgroup, you have to create an ACL \emph{role} (you have to choose
an ASCII name for it). In this example, the group is called
``project-x''.

\begin{verbatim}
$ wendzelnntpadm addaclrole project-x
Role project-x does not exists: okay.
done.
\end{verbatim}

\hypertarget{deleting-an-acl-role}{%
\paragraph{Deleting an ACL Role}\label{deleting-an-acl-role}}

You can delete an ACL role by using ``delaclrole'' instead of
``addaclrole'' like in the example above.

\hypertarget{connecting-and-disconnecting-users-withfrom-roles}{%
\subsubsection{Connecting and Disconnecting Users with/from
Roles}\label{connecting-and-disconnecting-users-withfrom-roles}}

To add (connect) or remove (disconnect) a user to/from a role, you need
to use the admin tool too.

\hypertarget{connecting-a-user-with-a-role}{%
\paragraph{Connecting a User with a
Role}\label{connecting-a-user-with-a-role}}

The second parameter (``project-x'') is the role name and the third
parameter (``developer1'') is the username. Here we add our three
developer users from the example above to the group project-x:

\begin{verbatim}
$ wendzelnntpadm roleuserconnect project-x developer1
Role project-x exists: okay.
User developer1 exists: okay.
Connecting role project-x with user developer1 ... done
done.
$ wendzelnntpadm roleuserconnect project-x developer2
Role project-x exists: okay.
User developer2 exists: okay.
Connecting role project-x with user developer2 ... done
done.
$ wendzelnntpadm roleuserconnect project-x developer3
Role project-x exists: okay.
User developer3 exists: okay.
Connecting role project-x with user developer3 ... done
done.
\end{verbatim}

\hypertarget{disconnecting-a-user-from-a-role}{%
\paragraph{Disconnecting a User from a
Role}\label{disconnecting-a-user-from-a-role}}

\begin{verbatim}
$ wendzelnntpadm roleuserdisconnect project-x developer1
Role project-x exists: okay.
User developer1 exists: okay.
Dis-Connecting role project-x from user developer1 ... done
done.
\end{verbatim}

\hypertarget{connecting-and-disconnecting-roles-withfrom-newsgroups}{%
\subsubsection{Connecting and Disconnecting Roles with/from
Newsgroups}\label{connecting-and-disconnecting-roles-withfrom-newsgroups}}

Even if a role is connected with a set of users, it is still useless
until you connect the role with a newsgroup.

\hypertarget{connecting-a-role-with-a-newsgroup}{%
\paragraph{Connecting a Role with a
Newsgroup}\label{connecting-a-role-with-a-newsgroup}}

To connect a role with a newsgroup, we have to use the command line tool
for a last time (the 2nd parameter is the role, and the 3rd parameter is
the name of the newsgroup). Here we connect our ``project-x'' role to
the two newsgroups ``project-x.discussion'' and ``project-x.support'':

\begin{verbatim}
$ wendzelnntpadm rolegroupconnect project-x project-x.discussion
Role project-x exists: okay.
Newsgroup project-x.discussion exists: okay.
Connecting role project-x with newsgroup project-x.discussion ... done
done.
$ wendzelnntpadm rolegroupconnect project-x project-x.support
Role project-x exists: okay.
Newsgroup project-x.support exists: okay.
Connecting role project-x with newsgroup project-x.support ... done
done.
\end{verbatim}

\hypertarget{disconnecting-a-role-from-a-newsgroup}{%
\paragraph{Disconnecting a Role From a
Newsgroup}\label{disconnecting-a-role-from-a-newsgroup}}

Disconnecting is done like in the example above but you have to use the
command ``rolegroup\textbf{dis}connect'' instead of
``rolegroupconnect''.

\hypertarget{listing-your-whole-acl-configuration-again}{%
\subsubsection{Listing Your Whole ACL Configuration
Again}\label{listing-your-whole-acl-configuration-again}}

Like mentioned before, you can use the command ``listacl'' to list your
whole ACL configuration (except the list of users that will be listed by
the command ``listusers'').

\begin{verbatim}
$ wendzelnntpadm listacl
List of roles in database:
Roles
-----
project-x

Connections between users and roles:
Role, User
----------
project-x, developer1
project-x, developer2
project-x, developer3

Username, Has access to group
-----------------------------
swendzel, alt.test

Role, Has access to group
-------------------------
project-x, project-x.discussion
project-x, project-x.support
done.
\end{verbatim}

\hypertarget{saving-time}{%
\paragraph{Saving time}\label{saving-time}}

As mentioned above, you can safe time by using roles. For instance, if
you add a new developer to the system, and the developer should have
access to the two groups ``project-x.discussion'' and
``project-x.support'', you do not have to assign the user to both (or
even more) groups by hand. Instead, you just add the user to the role
``project-x'' that is already connected to both groups.

If you want to give all developers access to the group
``project-x.news'', you also do not have to connect every developer with
the project. Instead, you just connect the role with the newsgroup, what
is one command instead of \(n\) commands. Of course, this time-saving
concept also works if you want to delete a user.

\hypertarget{hardening}{%
\subsection{Hardening}\label{hardening}}

Besides the already mentioned authentication, ACL and RBAC features, the
security of the server can be improved by putting WendzelNNTPd in a
\emph{chroot} environment or letting it run under an unprivileged user
account (the user then needs write access to
\emph{/var/spool/news/wendzelnntpd} and read access to
(\emph{/usr/local)/etc/wendzelnntpd.conf}). An unprivileged user under
Unix-like systems is also not able to create a listen socket on the
default NNTP port (119) since all ports up to 1023 are
usually\footnote{Some *nix-like systems may have a different range of
  privileged ports.} reserved. This means that the server should use a
port \(\geq\)1024 if it is started by a non-root user.

Please also note that WendzelNNTPd can be easily identified due to its
welcoming `banner' (desired code `200' message of NNTP). Tools such as
\texttt{nmap} provide rules to identify WendzelNNTPd and its version
this way. Theoretically, this could be changed by a slight code
modification (welcome message, HELP output and other components that
make the server identifiable). However, I do not recommend this as it is
just a form of `security by obscurity'.

\hypertarget{development-1}{%
\section{Development}\label{development-1}}

For development purposes you can start WendzelNNTPd on your host system.
If you are using an unsupported operating system or just not run
WenzdelNNTPd on your host system, you can use the provided Dockerfiles
to run WendzelNNTPd in a Docker container.

\hypertarget{initial-setup-1}{%
\subsubsection{Initial setup}\label{initial-setup-1}}

When you are on a UNIX-based system (like macOS) you can use the
provided make commands:

\begin{verbatim}
$ make docker-build
$ make docker-run
\end{verbatim}

To stop the Docker container you can use the following command:

\begin{verbatim}
$ make docker-stop
\end{verbatim}

If you are not on a UNIX-based system (like Windows) use the following
native docker commands:

\begin{verbatim}
$ docker build -f ./docker/Dockerfile -t wendzelnntpd:latest .
$ docker run --name wendzelnntpd --rm -it -p 118:118 -p 119:119 -p 563:563 -p 564:564 -v ${PWD}:/wendzelnntpd -v wendzelnntpd_data:/var/spool/news/wendzelnntpd wendzelnntpd:latest
\end{verbatim}

To stop the Docker container you can use the following command:

\begin{verbatim}
$ docker stop wendzelnntpd
\end{verbatim}

\hypertarget{test-new-code-1}{%
\subsubsection{Test new code}\label{test-new-code-1}}

The container is build without code included. The code is automatically
mounted as volume into the container. After each change of source code,
the application is compiled again.

\hypertarget{upgrading}{%
\section{Upgrading}\label{upgrading}}

\protect\hypertarget{Ch:Upgrade}{}{}

\hypertarget{upgrade-fom-version-2.1.y-to-2.2.x}{%
\subsection{Upgrade fom version 2.1.y to
2.2.x}\label{upgrade-fom-version-2.1.y-to-2.2.x}}

Please stop WendzelNNTPd and check the \emph{wendzelnntpd.conf}. There
is a new configuration style that breaks parts of the previous
configuration style (especially due to the introduction of
``connectors'').

\hypertarget{upgrade-from-version-2.1.x-to-2.1.y}{%
\subsection{Upgrade from version 2.1.x to
2.1.y}\label{upgrade-from-version-2.1.x-to-2.1.y}}

Same as upgrading from v.2.0.x to v.2.0.y, see Section
\protect\hyperlink{20xto20y}{6.4}.

~

~

\hypertarget{upgrade-from-version-2.0.x-to-2.1.y}{%
\subsection{Upgrade from version 2.0.x to
2.1.y}\label{upgrade-from-version-2.0.x-to-2.1.y}}

Please follow the upgrade instructions for upgrading from 2.0.x to 2.0.y
below. However, once you use cryptographic hashes in your
\emph{wendzelnntpd.conf}, your previous passwords will not work anymore,
i.e., you need to reset all passwords or deactivate the hashing feature.

~

~

\hypertarget{upgrade-from-version-2.0.x-to-2.0.y-20xto20y}{%
\subsection{Upgrade from version 2.0.x to 2.0.y
\{\#20xto20y\}}\label{upgrade-from-version-2.0.x-to-2.0.y-20xto20y}}

Stop WendzelNNTPd if it is currently running. Install WendzelNNTPd as
described but run \textbf{make upgrade} instead of \textbf{make
install}. Afterwards, start WendzelNNTPd again.

~

~

\hypertarget{upgrade-from-version-1.4.x-to-2.0.x}{%
\subsection{Upgrade from version 1.4.x to
2.0.x}\label{upgrade-from-version-1.4.x-to-2.0.x}}

\textbf{Acknowledgement:} I would like to thank Ann from Href.com for
helping a lot with finding out how to upgrade from 1.4.x to 2.0.x!

~

An upgrade from version 1.4.x was not foreseen due to the limited
available time I have for the development. However, here is a dirty
hack:

\hypertarget{preparation-step}{%
\subparagraph*{Preparation Step:}\label{preparation-step}}
\addcontentsline{toc}{subparagraph}{Preparation Step:}

You \textbf{need to create a backup of your existing installation
first}, at least everything from \textbf{/var/spool/news/wendzelnntpd}
as you will need all these files later. \textbf{Perform the following
steps on your own risk -- it is possible that they do not work on your
system as only two WendzelNNTPd installations were tested!}

\hypertarget{first-step}{%
\subparagraph*{First Step:}\label{first-step}}
\addcontentsline{toc}{subparagraph}{First Step:}

Install Wendzelnntpd-2.x on a Linux system (Windows is not supported
anymore). This requires some libraries and tools.

Under \emph{Ubuntu} they all come as packages:

\begin{verbatim}
$ sudo apt-get install libmysqlclient-dev libsqlite3-dev flex bison sqlite3
\end{verbatim}

Under \emph{CentOS} they come as packages as well:

\begin{verbatim}
$ sudo yum install make gcc bison flex sqlite-devel
\end{verbatim}

\emph{Other} operating systems should provide the same or similar
packages/ports.

Run \textbf{MYSQL=NO ./configure}, followed by \textbf{make}, and
\textbf{sudo make install}. This will compile, build and install
WendzelNNTPd without MySQL support as you only rely on SQLite3 from
v.1.4.x (and it would be significantly more difficult to bring the
SQLite database content to a MySQL database).

\hypertarget{second-step}{%
\subparagraph*{Second Step:}\label{second-step}}
\addcontentsline{toc}{subparagraph}{Second Step:}

Please make sure WendzelNNTPd-2 is configured in a way that we can
*hopefully* make it work with your own database file. Therefore, in the
configuration file \textbf{/usr/local/etc/wendzelnntpd.conf} make sure
that WendzelNNTPd uses sqlite3 instead of mysql:

\begin{verbatim}
database-engine sqlite3
\end{verbatim}

\hypertarget{third-step}{%
\subparagraph*{Third Step:}\label{third-step}}
\addcontentsline{toc}{subparagraph}{Third Step:}

Now comes the tricky part. The install command should have created
\textbf{/var/spool/news/wendzelnntpd/usenet.db}. However, it is an empty
usenet database file in the new format. Now REPLACE that file with the
file you use on your existing WendzelNNTPd installation, which uses the
old 1.4.x format. Also copy all of your old \textbf{cdp*} files and the
old \textbf{nextmsgid} file from your Windows system/from your backup
directory to \textbf{/var/spool/news/wendzelnntpd/}.

The following step is a very dirty hack but I hope it works for you. It
is not 100\% perfect as important table columns are then still of the
type `STRING' instead of the type `TEXT'!

Load the sqlite3 tool with your replaced database file:

\begin{verbatim}
$ sudo sqlite3 /var/spool/news/wendzelnntpd/usenet.db
\end{verbatim}

This will open the new file in editing mode. We now add the tables which
are not part of v.1.4.x to your existing database file. Therefore enter
the following commands:

\begin{verbatim}
CREATE TABLE roles (role TEXT PRIMARY KEY);
CREATE TABLE users2roles (username TEXT, role TEXT, PRIMARY KEY(username, role));
CREATE TABLE acl_users (username TEXT, ng TEXT, PRIMARY KEY(username, ng));
CREATE TABLE acl_roles (role TEXT, ng TEXT, PRIMARY KEY(role, ng));
.quit
\end{verbatim}

\hypertarget{fix-postings}{%
\subparagraph*{Fix Postings}\label{fix-postings}}
\addcontentsline{toc}{subparagraph}{Fix Postings}

You will probably see no post bodies right now if posts are requested by
your client. Therefore, switch into
\textbf{/var/spool/news/wendzelnntpd} and run (as superuser) this
command, it will replace the broken trailings with corrected ones:

\begin{verbatim}
$ cd /var/spool/news/wendzelnntpd
$ for filn in `/bin/ls cdp*`; do echo $filn; cat $filn | \
sed 's/\.\r/.\r\n/' > new;  num=`wc -l new| \
awk '{$minone=$1-1; print $minone}'` ; \
head -n $num new > $filn; done
$
\end{verbatim}

\hypertarget{last-step-checking-whether-it-works}{%
\subparagraph*{Last Step (Checking whether it
works!):}\label{last-step-checking-whether-it-works}}
\addcontentsline{toc}{subparagraph}{Last Step (Checking whether it
works!):}

First check, whether the database file is accepted at all:

\begin{verbatim}
$ sudo wendzelnntpadm listgroups
\end{verbatim}

It should list all your newsgroups

\begin{verbatim}
$ sudo wendzelnntpadm listusers
\end{verbatim}

It should list all existing users. Accordingly

\begin{verbatim}
$ sudo wendzelnntpadm listacl
\end{verbatim}

should list all access control entries (which will be empty right now
but if no error message appears, the related tables are now part of your
database file!).

~

Now start WendzelNNTPd via \textbf{sudo wendzelnntpd} and try to connect
with a NNTP client to your WendzelNNTPd and then try reading posts,
sending new posts and replying to these posts.

If this works, you can now run v2.x on 32bit and 64bit Linux :)
Congrats, you made it and chances are you are the second user who did
that. Let me know via e-mail if it worked for you.

\hypertarget{installation-1}{%
\section{Installation}\label{installation-1}}

This chapter provides a guide on how to install WendzelNNTPd 2.x.

\hypertarget{linuxnixbsd-1}{%
\subsection{Linux/*nix/BSD}\label{linuxnixbsd-1}}

To install WendzelNNTPd from source you need to download the provided
archive file (e.g., \emph{wendzelnntpd-2.0.0.tar.gz}) file.\footnote{Some
  *nix-like systems may have a different range of privileged ports.}
Extract it and run \textbf{./configure}. Please note that configure
indicates missing libraries and packages that you may first need to
install using the package system of your operating system.

\begin{verbatim}
$ tar -xzf wendzelnntpd-2.0.0.tgz
$ cd wendzelnntpd
$ ./configure
...
\end{verbatim}

\textbf{Please Note:} \emph{If you wish to compile WITHOUT MySQL or
WITHOUT SQlite support}, then run \textbf{MYSQL=NO ./configure} or
\textbf{SQLITE=NO ./configure}, respectively.

~

\textbf{Please Note:} \emph{For FreeBSD/OpenBSD/NetBSD/*BSD: There is no
MySQL support; you need to use SQLite (i.e., run
\texttt{MYSQL=NO\ ./configure}). Run \texttt{configure} as well as
\texttt{make} in the \texttt{bash} shell (under some BSDs you first need
to install \texttt{bash}).}

~

\textbf{Please Note:} \emph{If you wish to compile WITHOUT TLS support},
then run \textbf{TLS=NO ./configure}.

~

After \texttt{configure} has finished, run \textbf{make}:

\begin{verbatim}
$ make
...
\end{verbatim}

\hypertarget{generating-ssl-certifiates}{%
\subparagraph*{Generating SSL
certifiates}\label{generating-ssl-certifiates}}
\addcontentsline{toc}{subparagraph}{Generating SSL certifiates}

If you want to generate SSL certificates you can use the helper script:

\begin{verbatim}
$ sudo ./create_certificate \
    --environment letsencrypt \
    --email <YOUR-EMAIL> \\
    --domain <YOUR-DOMAIN>
\end{verbatim}

For the parameter \texttt{-\/-environment}, ``\emph{local}'' is also a
valid value. In that case, the certificate is generated only for usage
on localhost and is self-signed. After generating the certificate you
have to adjust \emph{wendzelnntpd.conf} (check
Section~\protect\hyperlink{network-settings}{{[}network-settings{]}}) to
activate TLS (configuration option \emph{enable-tls})). The paths for
certificate and server key can stay as they are.

~

\hypertarget{installing-wendzelnntpd}{%
\subparagraph*{Installing WendzelNNTPd}\label{installing-wendzelnntpd}}
\addcontentsline{toc}{subparagraph}{Installing WendzelNNTPd}

To install WendzelNNTPd on your system, you need superuser access. Run
\textbf{make install} to install it to the default location
\emph{/usr/local/*}.

\begin{verbatim}
$ sudo make install
\end{verbatim}

\textbf{Please Note (Upgrades):} Run \textbf{sudo make upgrade} instead
of \textbf{sudo make install} for an upgrade. Please
cf.~Section~\protect\hyperlink{Ch:Upgrade}{{[}Ch:Upgrade{]}}
(Upgrading).

\textbf{Please Note (MySQL):} \emph{If you plan to run MySQL}, then no
database was set up during `make install'. Please refer to
Section~\protect\hyperlink{Ch:Config}{{[}Ch:Config{]}} (Basic
Configuration) to learn how to generate the MySQL database.

\hypertarget{init-script-for-automatic-startup-1}{%
\subsubsection{Init Script for Automatic
Startup}\label{init-script-for-automatic-startup-1}}

There is an init script in the directory scripts/startup. It uses the
usual parameters like ``start'', ``stop'' and ``restart''.

\hypertarget{unofficial-note-mac-os-x-1}{%
\subsection{Unofficial Note: Mac OS
X}\label{unofficial-note-mac-os-x-1}}

A user reported WendzelNNTPd-2.0.0 is installable under Mac OS X 10.5.8.
The only necessary change was to add the flag
``\texttt{-arch\ x86\_64}'' to compile the code on a 64 bit system.
However, I never tried to compile WendzelNNTPd on a Mac.

\hypertarget{windows-1}{%
\subsection{Windows}\label{windows-1}}

Not supported.

\hypertarget{introduction-1}{%
\section{Introduction}\label{introduction-1}}

WendzelNNTPd is a tiny but easy to use Usenet server (NNTP server) for
Linux, *nix and BSD. The server is written in C. For security reasons,
it is compiled with stack smashing protection by default, if your
compiler supports that feature.

\hypertarget{features-1}{%
\subsection{Features}\label{features-1}}

\hypertarget{license-1}{%
\subsubsection{License}\label{license-1}}

WendzelNNTPd uses the GPLv3 license.

\hypertarget{database-abstraction-layer-1}{%
\subsubsection{Database Abstraction
Layer}\label{database-abstraction-layer-1}}

The server contains a database abstraction layer. Currently supported
database systems are SQlite3 and MySQL (and experimental PostgreSQL
support). New databases can be easily added.

\hypertarget{security-1}{%
\subsubsection{Security}\label{security-1}}

WendzelNNTPd contains different security features, the most important
features are probably Access Control Lists (ACLs) and the Role Based
Access Control (RBAC) system. ACL and RBAC are described in a own
chapter. WendzelNNTPd was probably the first Usenet server with support
for RBAC.

Another feature that was introduced by WendzelNNTPd (and later adopted
by other servers) are so-called ``invisible newsgroups'': If access
control is activated, a user without permission to access the newsgroup
will not be able to see the existence of the newsgroup. In case the user
knows about the existence of the newsgroup nevertheless, he will not be
able to post to or read from the newsgroup.

However, \textbf{please note} that the salting for password hashing is
using SHA-256, but with a global user-definable salt that is
concatenated with the username and password, rendering it less secure
than using unique random hashes per password. WendzelNNTPd does support
TLS v1.0 to v1.3 including STARTTLS!

\hypertarget{auto-prevention-of-double-postings-1}{%
\subsubsection{Auto-prevention of
double-postings}\label{auto-prevention-of-double-postings-1}}

In case a user sends a posting that lists the same newsgroup multiple
times within one post command's ``Newsgroups:'' header tag, the server
will add it only once to that newsgroup to save memory on the server and
the time of the readers.

\hypertarget{ipv6-1}{%
\subsubsection{IPv6}\label{ipv6-1}}

WendzelNNTPd supports IPv6. The server can listen on multiple IP
addresses as well as multiple ports.

\hypertarget{why-this-is-not-a-perfect-usenet-server-1}{%
\subsubsection{Why this is not a perfect Usenet
server}\label{why-this-is-not-a-perfect-usenet-server-1}}

WendzelNNTPd does not implement all NNTP commands, but the (most)
important ones. Another problem is that the regular expression library
used is not 100\% compatible with the NNTP matching in commands like
``XGTITLE''. Another limitation is that WendzelNNTPd cannot share
messages with other NNTP servers.

\hypertarget{contribute-1}{%
\subsection{Contribute}\label{contribute-1}}

See the \emph{CONTRIBUTE} file in the tarball.

\hypertarget{history-1}{%
\subsection{History}\label{history-1}}

The project started in 2004 under the name Xyria:cdpNNTPd, as part of
the Xyria project that also contained a fast DNS server, called
Xyria:DNSd. In 2007, I renamed it to WendzelNNTPd and stopped
development of Xyria:DNSd. Version 1.0.0 was released in 2007, version
2.0.0 in 2011. Since then I have primarily fixed reported bugs and added
minor features but the software is still maintained and smaller
advancements can still be expected. A detailed history can be found in
the \emph{HISTORY} file in the tarball. Fortunately, several people
contributed to the code and documentation, see \emph{AUTHORS} file.

\hypertarget{starting-and-running-wendzelnntpd-1}{%
\section{Starting and Running
WendzelNNTPd}\label{starting-and-running-wendzelnntpd-1}}

\hypertarget{starting-the-service-1}{%
\subsection{Starting the Service}\label{starting-the-service-1}}

Once your WendzelNNTPd installation has been configured, you can run the
server (in the default case you need superuser access to do that since
this is required to bind WendzelNNTPd to the default NNTP port 119) by
starting \textbf{/usr/local/sbin/wendzelnntpd}.

\begin{verbatim}
$ /usr/local/sbin/wendzelnntpd 
WendzelNNTPd: version 2.0.7 'Berlin'  - (Oct
26 2015 14:10:20 #2544) is ready.
\end{verbatim}

\textbf{Note (Daemon Mode):} If you want to run WendzelNNTPd as a
background daemon process on *nix-like operating systems, you should use
the parameter \textbf{-d}.

\hypertarget{stopping-and-restarting-the-service-1}{%
\subsection{Stopping and Restarting the
Service}\label{stopping-and-restarting-the-service-1}}

The server can be stopped by terminating its process:

\begin{verbatim}
$ pkill wendzelnntpd
\end{verbatim}

The server has a handler for the termination signal that allows to
safely shutdown using \textbf{pkill} or \textbf{kill}.

To restart the service, terminate and start the service.

\hypertarget{automating-start-stop-and-restart-1}{%
\subsubsection{Automating Start, Stop, and
Restart}\label{automating-start-stop-and-restart-1}}

The script \textbf{init.d\_script} in the directory
\emph{scripts/startup} of the tarball can be used to start, restart, and
stop WendzelNNTPd. It is a standard \emph{init.d} script for Linux
operating systems that can usually be copied to \emph{/etc/init.d} (it
must be executable).

\begin{verbatim}
$ cp scripts/startup/init.d_script /etc/init.d/wendzelnntpd
$ chmod +x /etc/init.d/wendzelnntpd
\end{verbatim}

\textbf{Note:} Please note that some operating systems use different
directories than \emph{/etc/init.d} or other startup script formats. In
such cases, the script works nevertheless but should simply be installed
to \emph{/usr/local/sbin} instead.

To start, stop, and restart WendzelNNTPd, the following commands can be
used afterwards:

\begin{verbatim}
$ /etc/init.d/wendzelnntpd start
Starting WendzelNNTPd ... done.
WendzelNNTPd: version 2.0.7 'Berlin'  - (Oct
26 2015 14:10:20 #2544) is ready.

$ /etc/init.d/wendzelnntpd restart
Stopping WendzelNNTPd ... done.
Starting WendzelNNTPd ... done.
WendzelNNTPd: version 2.0.7 'Berlin'  - (Oct
26 2015 14:10:20 #2544) is ready.

$ /etc/init.d/wendzelnntpd stop
Stopping WendzelNNTPd ... done.
\end{verbatim}

\hypertarget{administration-tool-wendzelnntpadm-1}{%
\subsection{Administration Tool
`wendzelnntpadm'}\label{administration-tool-wendzelnntpadm-1}}

Use the command line tool \textbf{wendzelnntpadm} to configure users,
roles and newsgroups of your WendzelNNTPd installation. To get an
overview of supported commands, run ``wendzelnntpadm help'':

\begin{verbatim}
$ wendzelnntpadm help
usage: wendzelnntpd <command> [parameters]
*** Newsgroup Administration:
 <listgroups>
 <addgroup | modgroup> <newsgroup> <posting-allowed-flag (y/n)>
 <delgroup> <newsgroup>
*** User Administration:
 <listusers>
 <adduser> <username> [<password>]
 <deluser> <username>
*** ACL (Access Control List) Administration:
 <listacl>
 <addacluser | delacluser> <username> <newsgroup>
 <addaclrole | delaclrole> <role>
 <rolegroupconnect | rolegroupdisconnect> <role> <newsgroup>
 <roleuserconnect | roleuserdisconnect> <role> <username>
\end{verbatim}

\hypertarget{creatinglistingdeleting-newsgroups-1}{%
\subsection{Creating/Listing/Deleting
Newsgroups}\label{creatinglistingdeleting-newsgroups-1}}

You can either list, create or delete newsgroups using
\textbf{wendzelnntpadm}.

\hypertarget{listing-existing-newsgroups-1}{%
\subsubsection{Listing existing
newsgroups}\label{listing-existing-newsgroups-1}}

\begin{verbatim}
$ wendzelnntpadm listgroups
Newsgroup, Low-, High-Value, Posting-Flag
-----------------------------------------
alt.test 10 1 y
mgmt.talk 1 1 y
secret.project-x 20 1 y
done.
\end{verbatim}

\hypertarget{creating-a-new-newsgroup-1}{%
\subsubsection{Creating a new
newsgroup}\label{creating-a-new-newsgroup-1}}

To create a new newsgroup run the following command:

\begin{verbatim}
$ wendzelnntpadm addgroup my.cool.group y
Newsgroup my.cool.group does not exist. Creating new group.
done.
\end{verbatim}

You can also change the ``posting allowed'' flag of a newsgroup:

\begin{verbatim}
$ wendzelnntpadm modgroup my.cool.group y
Newsgroup my.cool.group exists: okay.
done.
$ wendzelnntpadm modgroup my.cool.group n
Newsgroup my.cool.group exists: okay.
done.
\end{verbatim}

\hypertarget{deleting-a-newsgroup-1}{%
\subsubsection{Deleting a newsgroup}\label{deleting-a-newsgroup-1}}

\begin{verbatim}
$ wendzelnntpadm delgroup my.cool.group
Newsgroup my.cool.group exists: okay.
Clearing association class ... done
Clearing ACL associations of newsgroup my.cool.group... done
Clearing ACL role associations of newsgroup my.cool.group... done
Deleting newsgroup my.cool.group itself ... done
Cleanup: Deleting postings that do not belong to an existing newsgroup ... done
done.
\end{verbatim}

\hypertarget{user-accounts-administration-1}{%
\subsection{User Accounts
Administration}\label{user-accounts-administration-1}}

The easiest way to give only some people access to your server is to
create user accounts (please make sure you activated authentication in
your configuration file). You can add, delete and list all users.

\hypertarget{listing-users-and-passwords-1}{%
\subsubsection{Listing Users (and
Passwords)}\label{listing-users-and-passwords-1}}

This command always prints the (hashed) password of the users:

\begin{verbatim}
$ wendzelnntpadm listusers
Username, Password
------------------
developer1, wegerhgrhtrthjtzj
developer2, erghnrehhnht
manager1, wegergergrhth
manager2, thnthnrothnht
swendzel, lalalegergreg
swendzel2, 94j5z5jh5th
swendzel3, lalalalala
swendzel4, wegwegwegwegweg
done.
\end{verbatim}

\hypertarget{creating-a-new-user-1}{%
\subsubsection{Creating a new user}\label{creating-a-new-user-1}}

You can either enter the password as additional parameter (useful for
scripts that create users automatically) ...

\begin{verbatim}
$ wendzelnntpadm adduser UserName HisPassWord
User UserName does currently not exist: okay.
done.
\end{verbatim}

... or you can type it using the prompt (in this case the input is
shadowed):

\begin{verbatim}
$ wendzelnntpadm adduser UserName2
Enter new password for this user (max. 100 chars):
User UserName2 does currently not exist: okay.
done.
\end{verbatim}

\textbf{Please Note:} A password must include at least 8 characters and
may not include more than 100 characters.

\hypertarget{deleting-an-existing-user-1}{%
\subsubsection{Deleting an existing
user}\label{deleting-an-existing-user-1}}

\begin{verbatim}
$ wendzelnntpadm deluser UserName2
User UserName2 exists: okay.
Clearing ACL associations of user UserName2... done
Clearing ACL role associations of user UserName2... done
Deleting user UserName2 from database ... done
done.
\end{verbatim}

\hypertarget{access-control-list-administration-in-case-the-standard-nntp-authentication-is-not-enough-1}{%
\subsection{Access Control List Administration (in case the standard
NNTP authentication is not
enough)}\label{access-control-list-administration-in-case-the-standard-nntp-authentication-is-not-enough-1}}

Welcome to the advanced part of WendzelNNTPd. WendzelNNTPd includes a
powerful role based access control system. You can either only use
normal access control lists where you can configure which user will have
access to which newsgroup. Or you can use the advanced role system: You
can add users to roles (e.g., the user ``boss99'' to the role
``management'') and give a role access to a group (e.g., role
``management'' shall have access to ``discuss.management'').

\textbf{Note:} Please note that you must activate the ACL feature in
your configuration file to use it.

\textbf{Note:} To see \emph{all} data related to the ACL subsystem of
your WendzelNNTPd installation, simply use ``wendzelnntpadm listacl''.

\hypertarget{invisible-newsgroups-1}{%
\subsubsection{Invisible Newsgroups}\label{invisible-newsgroups-1}}

WendzelNNTPd includes a feature called ``Invisible Newsgroups'' which
means that a user without access to a newsgroup will neither see the
newsgroup in the list of newsgroups, nor will he be able to post to such
a newsgroup or will be able to read it.

\hypertarget{simple-access-control-1}{%
\subsubsection{Simple Access Control}\label{simple-access-control-1}}

We start with the simple access control component where you can define
which user will have access to which newsgroup.

\hypertarget{giving-a-user-access-to-a-newsgroup-1}{%
\paragraph{Giving a user access to a
newsgroup}\label{giving-a-user-access-to-a-newsgroup-1}}

\begin{verbatim}
$ wendzelnntpadm addacluser swendzel alt.test
User swendzel exists: okay.
Newsgroup alt.test exists: okay.
done.
$ wendzelnntpadm listacl
List of roles in database:
Roles
-----

Connections between users and roles:
Role, User
----------

Username, Has access to group
-----------------------------
swendzel, alt.test

Role, Has access to group
-------------------------
done.
\end{verbatim}

\hypertarget{removing-a-users-access-to-a-newsgroup-1}{%
\paragraph{Removing a user's access to a
newsgroup}\label{removing-a-users-access-to-a-newsgroup-1}}

\begin{verbatim}
$ wendzelnntpadm delacluser swendzel alt.test
User swendzel exists: okay.
Newsgroup alt.test exists: okay.
done.
\end{verbatim}

\hypertarget{adding-and-removing-acl-roles-1}{%
\subsubsection{Adding and Removing ACL
Roles}\label{adding-and-removing-acl-roles-1}}

If you have many users, some of them should have access to the same
newsgroup (e.g., the developers of a new system should have access to
the development newsgroup of the system). With roles you do not have to
give every user explicit access to such a group. Instead you add the
users to a role and give the role access to the group. (One advantage is
that you can easily give the complete role access to another group with
only one command instead of adding each of its users to the list of
people who have access to the new group).

In the following examples, we give the users ``developer1'',
``developer2'', and ``developer3'' access to the development role of
``project-x'' and connect their role to the newsgroups
``project-x.discussion'' and ``project-x.support''. To do so, we create
the three users and the two newsgroups first:

\begin{verbatim}
$ wendzelnntpadm adduser developer1
Enter new password for this user (max. 100 chars):
User developer1 does currently not exist: okay.
done.
$ wendzelnntpadm adduser developer2
Enter new password for this user (max. 100 chars):
User developer2 does currently not exist: okay.
done.
$ wendzelnntpadm adduser developer3
Enter new password for this user (max. 100 chars):
User developer3 does currently not exist: okay.
done.

$ wendzelnntpadm addgroup project-x.discussion y
Newsgroup project-x.discussion does not exist. Creating new group.
done.
$ wendzelnntpadm addgroup project-x.support y
Newsgroup project-x.support does not exist. Creating new group.
done.
\end{verbatim}

\hypertarget{creating-an-acl-role-1}{%
\paragraph{Creating an ACL Role}\label{creating-an-acl-role-1}}

Before you can add users to a role and before you can connect a role to
a newsgroup, you have to create an ACL \emph{role} (you have to choose
an ASCII name for it). In this example, the group is called
``project-x''.

\begin{verbatim}
$ wendzelnntpadm addaclrole project-x
Role project-x does not exists: okay.
done.
\end{verbatim}

\hypertarget{deleting-an-acl-role-1}{%
\paragraph{Deleting an ACL Role}\label{deleting-an-acl-role-1}}

You can delete an ACL role by using ``delaclrole'' instead of
``addaclrole'' like in the example above.

\hypertarget{connecting-and-disconnecting-users-withfrom-roles-1}{%
\subsubsection{Connecting and Disconnecting Users with/from
Roles}\label{connecting-and-disconnecting-users-withfrom-roles-1}}

To add (connect) or remove (disconnect) a user to/from a role, you need
to use the admin tool too.

\hypertarget{connecting-a-user-with-a-role-1}{%
\paragraph{Connecting a User with a
Role}\label{connecting-a-user-with-a-role-1}}

The second parameter (``project-x'') is the role name and the third
parameter (``developer1'') is the username. Here we add our three
developer users from the example above to the group project-x:

\begin{verbatim}
$ wendzelnntpadm roleuserconnect project-x developer1
Role project-x exists: okay.
User developer1 exists: okay.
Connecting role project-x with user developer1 ... done
done.
$ wendzelnntpadm roleuserconnect project-x developer2
Role project-x exists: okay.
User developer2 exists: okay.
Connecting role project-x with user developer2 ... done
done.
$ wendzelnntpadm roleuserconnect project-x developer3
Role project-x exists: okay.
User developer3 exists: okay.
Connecting role project-x with user developer3 ... done
done.
\end{verbatim}

\hypertarget{disconnecting-a-user-from-a-role-1}{%
\paragraph{Disconnecting a User from a
Role}\label{disconnecting-a-user-from-a-role-1}}

\begin{verbatim}
$ wendzelnntpadm roleuserdisconnect project-x developer1
Role project-x exists: okay.
User developer1 exists: okay.
Dis-Connecting role project-x from user developer1 ... done
done.
\end{verbatim}

\hypertarget{connecting-and-disconnecting-roles-withfrom-newsgroups-1}{%
\subsubsection{Connecting and Disconnecting Roles with/from
Newsgroups}\label{connecting-and-disconnecting-roles-withfrom-newsgroups-1}}

Even if a role is connected with a set of users, it is still useless
until you connect the role with a newsgroup.

\hypertarget{connecting-a-role-with-a-newsgroup-1}{%
\paragraph{Connecting a Role with a
Newsgroup}\label{connecting-a-role-with-a-newsgroup-1}}

To connect a role with a newsgroup, we have to use the command line tool
for a last time (the 2nd parameter is the role, and the 3rd parameter is
the name of the newsgroup). Here we connect our ``project-x'' role to
the two newsgroups ``project-x.discussion'' and ``project-x.support'':

\begin{verbatim}
$ wendzelnntpadm rolegroupconnect project-x project-x.discussion
Role project-x exists: okay.
Newsgroup project-x.discussion exists: okay.
Connecting role project-x with newsgroup project-x.discussion ... done
done.
$ wendzelnntpadm rolegroupconnect project-x project-x.support
Role project-x exists: okay.
Newsgroup project-x.support exists: okay.
Connecting role project-x with newsgroup project-x.support ... done
done.
\end{verbatim}

\hypertarget{disconnecting-a-role-from-a-newsgroup-1}{%
\paragraph{Disconnecting a Role From a
Newsgroup}\label{disconnecting-a-role-from-a-newsgroup-1}}

Disconnecting is done like in the example above but you have to use the
command ``rolegroup\textbf{dis}connect'' instead of
``rolegroupconnect''.

\hypertarget{listing-your-whole-acl-configuration-again-1}{%
\subsubsection{Listing Your Whole ACL Configuration
Again}\label{listing-your-whole-acl-configuration-again-1}}

Like mentioned before, you can use the command ``listacl'' to list your
whole ACL configuration (except the list of users that will be listed by
the command ``listusers'').

\begin{verbatim}
$ wendzelnntpadm listacl
List of roles in database:
Roles
-----
project-x

Connections between users and roles:
Role, User
----------
project-x, developer1
project-x, developer2
project-x, developer3

Username, Has access to group
-----------------------------
swendzel, alt.test

Role, Has access to group
-------------------------
project-x, project-x.discussion
project-x, project-x.support
done.
\end{verbatim}

\hypertarget{saving-time-1}{%
\paragraph{Saving time}\label{saving-time-1}}

As mentioned above, you can safe time by using roles. For instance, if
you add a new developer to the system, and the developer should have
access to the two groups ``project-x.discussion'' and
``project-x.support'', you do not have to assign the user to both (or
even more) groups by hand. Instead, you just add the user to the role
``project-x'' that is already connected to both groups.

If you want to give all developers access to the group
``project-x.news'', you also do not have to connect every developer with
the project. Instead, you just connect the role with the newsgroup, what
is one command instead of \(n\) commands. Of course, this time-saving
concept also works if you want to delete a user.

\hypertarget{hardening-1}{%
\subsection{Hardening}\label{hardening-1}}

Besides the already mentioned authentication, ACL and RBAC features, the
security of the server can be improved by putting WendzelNNTPd in a
\emph{chroot} environment or letting it run under an unprivileged user
account (the user then needs write access to
\emph{/var/spool/news/wendzelnntpd} and read access to
(\emph{/usr/local)/etc/wendzelnntpd.conf}). An unprivileged user under
Unix-like systems is also not able to create a listen socket on the
default NNTP port (119) since all ports up to 1023 are
usually\footnote{Some *nix-like systems may have a different range of
  privileged ports.} reserved. This means that the server should use a
port \(\geq\)1024 if it is started by a non-root user.

Please also note that WendzelNNTPd can be easily identified due to its
welcoming `banner' (desired code `200' message of NNTP). Tools such as
\texttt{nmap} provide rules to identify WendzelNNTPd and its version
this way. Theoretically, this could be changed by a slight code
modification (welcome message, HELP output and other components that
make the server identifiable). However, I do not recommend this as it is
just a form of `security by obscurity'.

\hypertarget{upgrading-1}{%
\section{Upgrading}\label{upgrading-1}}

\protect\hypertarget{Ch:Upgrade}{}{}

\hypertarget{upgrade-fom-version-2.1.y-to-2.2.x-1}{%
\subsection{Upgrade fom version 2.1.y to
2.2.x}\label{upgrade-fom-version-2.1.y-to-2.2.x-1}}

Please stop WendzelNNTPd and check the \emph{wendzelnntpd.conf}. There
is a new configuration style that breaks parts of the previous
configuration style (especially due to the introduction of
``connectors'').

\hypertarget{upgrade-from-version-2.1.x-to-2.1.y-1}{%
\subsection{Upgrade from version 2.1.x to
2.1.y}\label{upgrade-from-version-2.1.x-to-2.1.y-1}}

Same as upgrading from v.2.0.x to v.2.0.y, see Section
\protect\hyperlink{20xto20y}{1.4}.

~

~

\hypertarget{upgrade-from-version-2.0.x-to-2.1.y-1}{%
\subsection{Upgrade from version 2.0.x to
2.1.y}\label{upgrade-from-version-2.0.x-to-2.1.y-1}}

Please follow the upgrade instructions for upgrading from 2.0.x to 2.0.y
below. However, once you use cryptographic hashes in your
\emph{wendzelnntpd.conf}, your previous passwords will not work anymore,
i.e., you need to reset all passwords or deactivate the hashing feature.

~

~

\hypertarget{upgrade-from-version-2.0.x-to-2.0.y-20xto20y-1}{%
\subsection{Upgrade from version 2.0.x to 2.0.y
\{\#20xto20y\}}\label{upgrade-from-version-2.0.x-to-2.0.y-20xto20y-1}}

Stop WendzelNNTPd if it is currently running. Install WendzelNNTPd as
described but run \textbf{make upgrade} instead of \textbf{make
install}. Afterwards, start WendzelNNTPd again.

~

~

\hypertarget{upgrade-from-version-1.4.x-to-2.0.x-1}{%
\subsection{Upgrade from version 1.4.x to
2.0.x}\label{upgrade-from-version-1.4.x-to-2.0.x-1}}

\textbf{Acknowledgement:} I would like to thank Ann from Href.com for
helping a lot with finding out how to upgrade from 1.4.x to 2.0.x!

~

An upgrade from version 1.4.x was not foreseen due to the limited
available time I have for the development. However, here is a dirty
hack:

\hypertarget{preparation-step}{%
\subparagraph*{Preparation Step:}\label{preparation-step}}
\addcontentsline{toc}{subparagraph}{Preparation Step:}

You \textbf{need to create a backup of your existing installation
first}, at least everything from \textbf{/var/spool/news/wendzelnntpd}
as you will need all these files later. \textbf{Perform the following
steps on your own risk -- it is possible that they do not work on your
system as only two WendzelNNTPd installations were tested!}

\hypertarget{first-step}{%
\subparagraph*{First Step:}\label{first-step}}
\addcontentsline{toc}{subparagraph}{First Step:}

Install Wendzelnntpd-2.x on a Linux system (Windows is not supported
anymore). This requires some libraries and tools.

Under \emph{Ubuntu} they all come as packages:

\begin{verbatim}
$ sudo apt-get install libmysqlclient-dev libsqlite3-dev flex bison sqlite3
\end{verbatim}

Under \emph{CentOS} they come as packages as well:

\begin{verbatim}
$ sudo yum install make gcc bison flex sqlite-devel
\end{verbatim}

\emph{Other} operating systems should provide the same or similar
packages/ports.

Run \textbf{MYSQL=NO ./configure}, followed by \textbf{make}, and
\textbf{sudo make install}. This will compile, build and install
WendzelNNTPd without MySQL support as you only rely on SQLite3 from
v.1.4.x (and it would be significantly more difficult to bring the
SQLite database content to a MySQL database).

\hypertarget{second-step}{%
\subparagraph*{Second Step:}\label{second-step}}
\addcontentsline{toc}{subparagraph}{Second Step:}

Please make sure WendzelNNTPd-2 is configured in a way that we can
*hopefully* make it work with your own database file. Therefore, in the
configuration file \textbf{/usr/local/etc/wendzelnntpd.conf} make sure
that WendzelNNTPd uses sqlite3 instead of mysql:

\begin{verbatim}
database-engine sqlite3
\end{verbatim}

\hypertarget{third-step}{%
\subparagraph*{Third Step:}\label{third-step}}
\addcontentsline{toc}{subparagraph}{Third Step:}

Now comes the tricky part. The install command should have created
\textbf{/var/spool/news/wendzelnntpd/usenet.db}. However, it is an empty
usenet database file in the new format. Now REPLACE that file with the
file you use on your existing WendzelNNTPd installation, which uses the
old 1.4.x format. Also copy all of your old \textbf{cdp*} files and the
old \textbf{nextmsgid} file from your Windows system/from your backup
directory to \textbf{/var/spool/news/wendzelnntpd/}.

The following step is a very dirty hack but I hope it works for you. It
is not 100\% perfect as important table columns are then still of the
type `STRING' instead of the type `TEXT'!

Load the sqlite3 tool with your replaced database file:

\begin{verbatim}
$ sudo sqlite3 /var/spool/news/wendzelnntpd/usenet.db
\end{verbatim}

This will open the new file in editing mode. We now add the tables which
are not part of v.1.4.x to your existing database file. Therefore enter
the following commands:

\begin{verbatim}
CREATE TABLE roles (role TEXT PRIMARY KEY);
CREATE TABLE users2roles (username TEXT, role TEXT, PRIMARY KEY(username, role));
CREATE TABLE acl_users (username TEXT, ng TEXT, PRIMARY KEY(username, ng));
CREATE TABLE acl_roles (role TEXT, ng TEXT, PRIMARY KEY(role, ng));
.quit
\end{verbatim}

\hypertarget{fix-postings}{%
\subparagraph*{Fix Postings}\label{fix-postings}}
\addcontentsline{toc}{subparagraph}{Fix Postings}

You will probably see no post bodies right now if posts are requested by
your client. Therefore, switch into
\textbf{/var/spool/news/wendzelnntpd} and run (as superuser) this
command, it will replace the broken trailings with corrected ones:

\begin{verbatim}
$ cd /var/spool/news/wendzelnntpd
$ for filn in `/bin/ls cdp*`; do echo $filn; cat $filn | \
sed 's/\.\r/.\r\n/' > new;  num=`wc -l new| \
awk '{$minone=$1-1; print $minone}'` ; \
head -n $num new > $filn; done
$
\end{verbatim}

\hypertarget{last-step-checking-whether-it-works}{%
\subparagraph*{Last Step (Checking whether it
works!):}\label{last-step-checking-whether-it-works}}
\addcontentsline{toc}{subparagraph}{Last Step (Checking whether it
works!):}

First check, whether the database file is accepted at all:

\begin{verbatim}
$ sudo wendzelnntpadm listgroups
\end{verbatim}

It should list all your newsgroups

\begin{verbatim}
$ sudo wendzelnntpadm listusers
\end{verbatim}

It should list all existing users. Accordingly

\begin{verbatim}
$ sudo wendzelnntpadm listacl
\end{verbatim}

should list all access control entries (which will be empty right now
but if no error message appears, the related tables are now part of your
database file!).

~

Now start WendzelNNTPd via \textbf{sudo wendzelnntpd} and try to connect
with a NNTP client to your WendzelNNTPd and then try reading posts,
sending new posts and replying to these posts.

If this works, you can now run v2.x on 32bit and 64bit Linux :)
Congrats, you made it and chances are you are the second user who did
that. Let me know via e-mail if it worked for you.

%\addcontentsline{toc}{chapter}{Index}
%\printindex

\end{document}
